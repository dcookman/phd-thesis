\chapter{The SMELLIE Calibration System}\label{chap:smellie_hardware}
\epigraph{\textit{There's a certain Slant of light,\\
Winter Afternoons ---\\
That oppresses, like the Heft\\
Of Cathedral Tunes ---}}{\textsc{Emily Dickinson}}
{
    \color{blue}

\begin{itemize}
    \item Basic principle for how SMELLIE works: firing collimated laser light into detector to observe scattering events.
    \item Analysis will measure and monitor scattering in a detector with changing optics.
    \item One can try and measure some component of this: the cross-section/scattering length versus wavelength and time, and/or the relative scattering length versus wavelength and time.
\end{itemize}
[1 page]
\section{The SMELLIE Hardware}
\begin{itemize}
    \item Describe the existing hardware, post-upgrade made in Summer 2022. For pre-upgrade hardware, can simply cite previous SMELLIE theses. This includes the path of light into the detector, as well as the path of the trigger signal.
    \item Make sure to mention explicitly these upgrades: Tony Zummo's fix to the TUBii trigger logic, as well as the addition of the VFA, updated MPU, and modified trigger window. Make sure to motivate why these updates were made.
\end{itemize}
[7 pages]
\section{Software for SMELLIE Data-taking}
\begin{itemize}
    \item Can be brief here! Little has changed since previous theses, so can mostly just summarise and cite.
    \item Server running on SNODROP machine, which converts high-level commands into low-level ones that the hardware can interpret.
    \item Run plan files written in JSON handed to ORCA which then sends relevant commands to SNODROP which fires as appropriate.
    \item Operator interacts with ORCA to perform SMELLIE calibration runs.
    \item After SMELLIE data taken, run description file created, containing metadata about the run conditions, used in analysis.
\end{itemize}
[2 pages]
\section[Commissioning SMELLIE in the Scintillator Phase]{Commissioning SMELLIE in the\\ Scintillator Phase}
\begin{itemize}
    \item Explain why commissioning of SMELLIE is needed: Need to confirm that SMELLIE is working as expected; determine intensity "set-points" for different use cases.
    \item Commissioning originally performed by Esther and JeffL back in the water phase; explain why this needed to be re-done for both the scintillator phase and after the hardware upgrades.
    \item No need to describe the Tesseract in detail here - that can be in Jeff L's thesis. But, I do want to show the results of both commissioning campaigns in scintillator-fill, one before the new hardware was added, and one after.
\end{itemize}
[5 pages]
[15 PAGES TOTAL]
}