\chapter{Solar Oscillation Analysis}\label{chap:solar_osc_analysis}
\epigraph{\textit{Driving out into the Sun\\Let the ultraviolet cover me up\\Looking for a Creation Myth\\Ended up with a pair of black lips}}{\textit{This is the End}\\ \textsc{Phoebe Bridgers}}
Measuring the ``solar'' neutrino oscillation parameters \dmsq{} and \tonetwo{} is one of the principal aims of the SNO+ detector during the scintillator-phase. There are, in fact, two complementary methods of measuring these parameters: the oscillations of anti-neutrinos from terrestrial nuclear reactors, and the oscillations of neutrinos from the Sun.

This chapter focuses on the latter approach, using \beight{} neutrinos coming from the Sun to measure the solar oscillation parameters. An initial background-free study was performed by Javi Caravaca~\cite{}, % TODO: Javi's tech note
which demonstrated that it was indeed possible to make such a measurement in the detector. The work in this chapter builds on substantially from that analysis. This chapter also draws on the associated reactor anti-neutrino analysis built by Iwan Morton-Blake~\cite{}, %TODO: Iwan's thesis
and more broadly from the general techniques used in the \onbb{} analysis of Tereza Kroupova~\cite{} and Jack Dunger~\cite{}.% TODO: Tereza & Jack's thesis

This chapter begins by explaining how it is possible to measure the solar oscillation parameters via \beight{} events. Then, the framework used to perform the analysis is then explained: that of a \textit{Bayesian Analysis using Markov Chain Monte Carlo techniques}. After the method has been described, the dataset upon which the analysis is performed is then introduced. The results and associated validation are then given. Given these results, a projection is then made for the expected sensitivity to \tonetwo{} as a function of livetime.


\section{Analysis Methodology}\label{sec:analysis_method}
\subsection{Observational Principle}
How can we measure neutrino oscillation parameters via solar neutrinos in the SNO+ detector? As discussed in~\label{}, % link to theory section on solar oscillations
it is possible to detect all flavours of neutrino through elastic scattering with electrons in the detector. If this interaction was purely neutral-current, then there would be no way of telling the flavour-state of an interacting neutrino. However, electron neutrinos are able to interact through an additional charged-current mode. This modifies the cross-section for electron neutrinos, and means that as the survival probability for electron neutrinos generated from the Sun, $P_{ee}$, is modified, the interaction probability of neutrinos with the detector will also.

Of course, we do not directly measure neutrino energies in the detector --- only the associated scattered electron. If there were no correlation between the observed electron energy and its associated neutrino, then the only effect of neutrino oscillations would be to change the overall observed rate of events due to this process. There would be no change in the shape of the event's energy spectrum, even though neutrino oscillations are a function of neutrino energy. Fortunately, there is some dependence of the neutrino's energy on that of the scattered electron. This dependence can be seen in Fig.~\ref{fig:nu_elec_energy_dependence} for \beight{} electron neutrinos interacting in SNO+. % Expected analytical dependence?
As can be seen, the dependence is weak, and comes mostly from basic energy conservation: If one observes a \SI{10}{\MeV} electron event in the detector, it can't reasonably have come from a \SI{5}{\MeV} neutrino.

\begin{figure}
    \centering
    % \includegraphics[]{}
    \caption[]{}
    \label{fig:nu_elec_energy_dependence}
\end{figure}

\begin{figure}
    \centering
    % \includegraphics[]{}
    \caption[]{}
    \label{fig:nu_elec_energy_dependence2}
\end{figure}

In Fig.~\ref{fig:nu_elec_energy_dependence2} we can see the impact each physical process has on the energy spectrum that we eventually observe. We start with a broad energy distribution of \beight{} electron neutrinos generated in the Sun. These neutrinos then oscillate their flavour state as they propagate to the detector, in an energy-dependent manner. When a (tiny) fraction of these neutrinos interact with the electrons in our detector, there is both an energy- and flavour-dependence on the cross-section. The scattered electrons gain a kinetic energy with some mild dependence on the inciting neutrino's energy, which is then measured by the detector to within some energy resolution.

Let us now consider the dependence of $P_{ee}$ on the individual neutrino oscillation parameters. Recall from Eq.~\ref{} %
that, after considering matter-induced oscillations due to neutrinos passing through the Sun and possibly the Earth, $P_{ee} = P_{ee}\left(\tan2\theta^{M}_{12}, \sin\theta^{M}_{13}, \Delta m^{2}_{21,M}\right) = P_{ee}\left(\theta_{12}, \theta_{13}, \Delta m^{2}_{12}, \Delta m^{2}_{13}\right)$. Fig.~\ref{fig:pee_osc_param_dependence} shows the dependence of each of these four oscillation parameters on $P_{ee}(E)$. We can see that in reality only the two parameters \dmsq{} and \tonetwo{} have a substantial impact on $P_{ee}(E)$ and hence the observed electron energy spectrum. Because of this, for this analysis we will only ever vary these two oscillation parameters, and keep $\theta_{13}$ and $\Delta m^{2}_{13}$ at their current global fit values\footnote{We use the global fit results excluding Super-Kamiokande's atmospheric data, and assuming normal ordering of the neutrino mass hierarchy. This choice has a tiny impact on the magnitudes of these two fixed parameters, the main impact being the sign of $\Delta m^{2}_{13}$.} % Confirm that mass hierarchy has no impact on Pee.
of $\sin^{2}\theta_{13} = 0.0222$ and $\Delta m^{2}_{13} = +\SI{2.515e-3}{\eV\squared}$~\cite{}.% nufit citation

\begin{figure}
    \centering
    % \includegraphics[]{}
    \caption[]{}
    \label{fig:pee_osc_param_dependence}
\end{figure}

\subsection{Background Processes}
Sadly, elastically-scattered electrons from \beight{} neutrinos are not the only events we see in the SNO+ detector during the scintillator phase. There are a number of background processes that our signal must compete against. Below a reconstructed energy of $\sim\SI{2}{\MeV}$, it is known that various backgrounds completely swamp any possible \beight{} signal, and so for this analysis we only consider processes that can generate reconstructed energies of at least $E_{\textrm{min}} = \SI{2}{\MeV}$. The following subsections explain each of these backgrounds, as well as methods that have been used to mitigate them as much as possible.

\subsubsection{Internal Uranium- and Thorium-Chain Backgrounds}
Although every effort has been made to make the scintillator cocktail that fills SNO+ to be as radio-pure as possible, there inevitably remain trace amounts of the radioactive isotopes that derive from the decay chains of the \ce{^{238}U} and \ce{^{232}Th} isotopes. Fig.~\ref{fig:u_th_decay_chains} shows these two decay chains. Fortunately, only a fraction of the radioactive isotopes in these chains actually are capable of generating events in the detector with energies above $E_{\textrm{min}}$: these have been highlighted in Fig.~\ref{fig:u_th_decay_chains} in gold.

\begin{figure}
    \centering
    % \includegraphics[]{}
    \caption[]{}
    \label{fig:u_th_decay_chains}
\end{figure}

Of particular note are the decays of \ce{^{212}Bi} and \ce{^{214}Bi}. Both are capable of either $\alpha\textrm{-}$ or $\beta\textrm{-decays}$ to \ce{Tl} or \ce{Po} isotopes, respectively. For the former, it is the subsequent $\beta$-decay of the \ce{Tl} that can have a reconstructed energy above $E_{\textrm{min}}$. For the latter, the \ce{Bi} decay is the part of the pair of decays that can lie above $E_{\textrm{min}}$. Although the $\alpha$-decays here certainly have Q-values well above \SI{2}{\MeV}, the liquid scintillator quenches the observed energy well below $E_{\mathrm{min}}$. The so-called ``\ce{Bi-Po}'' decays are particularly special because the lifetimes of \ce{^{212}Po} and \ce{^{214}Po} are \SI{300}{\nano\second} and \SI{164}{\micro\second}, respectively, which are short enough to allow for highly-effective coincidence tagging.

There are two classes of \ce{Bi-Po} event in the detector: ``out-of-window'' events for which the \ce{Bi} and \ce{Po} occur in separate event windows, and ``in-window'' events whereby the \ce{Bi} and \ce{Po} occur within the same event window. These lead to two distinct strategies for tagging these kinds of events. For out-of-window \ce{Bi-Po}s, we look for a delayed coincidence of two events. Using the tagging algorithm suggested in~\cite{} % Jeanne's BiPo tagging DocDB
as a starting point, the chosen procedure was as follows. There must be two events that trigger the detector within \SI{4}{\micro\second} of one another, and both have a valid \texttt{scintFit} position reconstruction within \SI{2}{\metre} of one another. The delayed candidate event must also have at least 100 cleaned PMT hits. %Must ensure cleaned nhits has been explained earlier
This very broad coincidence tagging procedure was designed to ensure that the cut was as \textit{efficient} in tagging (and hence, rejecting) \ce{Bi-Po}s as possible, whilst negligibly impacting the solar signal. This is in contrast to the cuts chosen by Rafael Hunt-Stokes in~\cite{}, % Raf's BiPo tagging tech note on DocDB
which try and obtain a highly \textit{pure} sample of \ce{Bi-Po} tags.

Of course, the above delayed coincidence procedure cannot catch any of the in-window \ce{Bi-Po} events. For these, we use a different approach. Because two decays happened in the same event window, we expect to see two distinct peaks in the event's time residual spectrum. In order to look for this event topology, a likelihood-ratio classifier was run over events, first developed by Eric Marzec~\cite{} % Eric's tech note on DocDB
and re-coordinated for the \SI{2.2}{\gram\per\litre} LABPPO scintillator optics by Ziping Ye~\cite{}. % Ziping's presentation on DocDB
This classifier calculates the likelihood ratio between the null hypothesis of a \onbb{} event (a proxy in this analysis for single-site events such as our \beight{} signal) and the alternative hypothesis of an in-window \ce{Bi-Po} event. The more negative the value of the result, \texttt{alphabeta212}, the greater the evidence there is for rejecting the null hypothesis of a single-site event. Events with $\texttt{alphabeta212} < 0$ % WRITE FINAL CHOICE OF CUT!
were then rejected.

Combining both out-of-window and in-window \ce{Bi-Po} tagging, the impact on \ce{^{212}Bi-Po}, \ce{^{214}Bi-Po}, and \beight{} $\nu_e$ events can be seen in Fig.~\ref{fig:bipo_tagging_efficiency}. We consider here only events that pass all other cuts used in this analysis: see section~\ref{} % link to cuts subsection
for the specifics of the cuts used. Because of the different lifetimes of the decays, \ce{^{214}Bi-Po} decays predominantly fall out-of-window whilst \ce{^{212}Bi-Po} events are typically in-window. This explains why the out-of-window tagging is substantially better at cutting \ce{^{214}Bi-Po} decays, whereas the in-window tagging far better tags \ce{^{212}Bi-Po} decays. Overall, within the analysis region of interest, the two combined cuts are able to tag TODO\% % CALCULATE
of \ce{^{214}Bi-Po} triggered events, TODO\% % CALCULATE
of \ce{^{212}Bi-Po} triggered events, whilst retaining TODO\% % CALCULATE
of \beight{} $\nu_e$ signal events.

\begin{figure}
    \centering
    % \includegraphics[]{}
    \caption[]{}
    \label{fig:bipo_tagging_efficiency}
\end{figure}

\subsubsection{Cosmogenic Isotopes}


     \begin{itemize}
        % \item Start with observational principle: changes in oscillation parameters lead to change in the energy spectrum of neutrinos elastically-scattering in the detector, and hence a change in the observed energy spectra of elastically-scattered electrons within the detector. Demonstrate basic impact on modifying $\Delta m^{2}_{21}$ and $\theta_{12}$.
        \item Want to maximise the sensitivity to measuring these two parameters. Energy spectrum of signal is background-free above $\sim\SI{5}{\MeV}$, but rate is substantially larger at lower energies. If one can minimise backgrounds at the lower energies, then there should be hopes of obtaining a measurement with greater precision!
        \item Set up analysis approach: a Bayesian analysis using MCMC. Explain why this was chosen at a high level first: allows for us to perform a relatively complex analysis with multidimensional PDFs, numerous backgrounds, systematics, constraints on parameters, all whilst allowing us to obtain well-defined measures of uncertainty on our final results.
        \item Test statistic: binned extended log-likelihood. Explain why this is fundamentally the ``right''  test statistic to use. Allows for handling of constraints and systematics.
        \item Give an overview of how an MCMC works. Key idea: exploring parameter space in such a way as to reproduce posterior density distribution. Clever! Helps to avoid ``curse of dimensionality'' that often arises in fits with numerous parameters. Must also explain how we work in a fundamentally Bayesian, not frequentist, statistical approach.
        \item Explain background processes we'll be dealing with: can be distinguished by event topology, energy spectrum, and position distribution.
        \item Explain cuts that I am using for the analysis. Demonstrate how these attempt to maximise our signal sensitivity. Leads nicely into choice of 2D fit in both energy and radius (cubed).
        \item Describe implementation of neutrino oscillations within the fitting procedure: flat priors on the oscillation parameters, but a strong constraint from the solar flux. Explain choices of constraint that are possible. MCMC varies oscillation parameters and flux scaling factor, which then modifies the solar signal PDFs through a de-facto systematic that is a function of true neutrino energy, a 3\textsuperscript{rd} ``bookkeeping'' dimension in the signal's PDFs only. Neutrino oscillations simulated via PSelmaa, which accounts for MSW effect in both the Sun and the Earth. For computational speed, at run-time we actually use a lookup table with linear interpolation for the survival probability as a function of parameters.
        \item Implementation of systematics: we handle them generally as linear transformations acting on the vector of bin data. Clever! Which systematics do we expect to be particularly important for this analysis? Well, mismodelling in detector optics etc. can lead to changes in the measured energy spectrum of processes, which can be decomposed into an energy scale term and an energy smearing term to first order. Systematics also possible in the radial dimension (expected to be less important?) 
        
     \end{itemize}

\section{Analysis on Scintillator-Phase data}
\begin{itemize}
    \item Description of dataset chosen for analysis: full-scint data that satisfies the ``gold'' list of run selection requirements, between 1\textsuperscript{st} June 2022 and March 2023. Starting date chosen to ensure radon levels have stabilised within the centre of the detector.
    \item Impact of cuts on data and MC. Show tables (the full details maybe in an appendix) indicating this.
    \item Describe the constraints chosen to apply to the fit, and why they can be justified.
    \item Running \& validation of MCMC fitting. Show plots of parameter values versus step, to demonstrate that the step sizes have been tuned sufficiently. Show auto-correlation plots, to motivate a sensible ``burn-in'' size. Show posterior density plots for each nuisance parameter, to check that they all look sensible and have sufficient statistics. Show plot of correlation coefficients between parameters, and look at any correlations that are particularly interesting.
    \item Look at the data versus MC plot in energy, radius, and both. Is there a good fit to data? Any clear disagreements?
    \item Show 2D contour plot for oscillation parameter posterior density. Note salient features. Show 1D posterior densities for each oscillation parameter. Derive measurement result for $\theta_{12}$.
    \item Show impact of modifying certain constraints on the final results of the measurement of $\theta_{12}$.
\end{itemize}

\section{Sensitivity Projections}


\section{Conclusions}