%!TEX root = ../thesis.tex
%*******************************************************************************
%*********************************** First Chapter *****************************
%*******************************************************************************

\chapter{The Theory of Neutrino Physics}\label{chap:theory}
\setlength{\epigraphwidth}{.45\textwidth}
\epigraph{\textit{Light\\Light\\The visible reminder of Invisible Light}}{\textit{The Rock}\\\textsc{T. S. Eliot}}
\setlength{\epigraphwidth}{.4\textwidth}
{
    \color{blue}
\section{The Standard Model and Neutrinos}
\subsection{A Brief Introduction to the Standard Model}
Covering how the SM works at the highest level, including:
\begin{itemize}
    \item Quantum Field Theory and the Lagrangian dynamical framework
    \item The connection between symmetries of a QFT model and its gauge fields that describe the model's forces
    \item The SM's fundamental symmetries, and associated forces, but ---
    \item Not (exactly) what we see ``normally''! The electromagnetic and weak forces appear distinct, and the weak gauge bosons have mass. To explain this, we need a further component, the Brout-Englert-Higgs (BEH) Mechanism. 
\end{itemize}
[2 pages total]
\subsection{Neutrinos within the Standard Model}
\begin{itemize}
    \item Basic description of where neutrinos fit into SM: 3 kinds of neutral fermion, the counterparts to the charged fermions. Interacts with the weak force only.
    \item Summary of the experimental evidence for this picture: mainly, the discovery of electron anti-neutrinos by Cowan and Reines, the muon neutrino by Lederman, Schwartz, and Steinberger, and the tau neutrino by the DONUT Collaboration. Further critical experiments include the first measurement of a neutrino's helicity by Goldhaber et al. as well as Danby et al.'s demonstration that $\nu_{\mu}$ are distinct from $\nu_{e}$.
    \item More detailed description, via Feynman diagrams, of the two fundamental modes of interaction by neutrinos with the weak force: charged- and neutral-current interactions. A brief mention of the quantitative theory that underlies description: Gashow, Salam, and Weinberg's Electroweak Theory. This explains not only the V--A structure of charged-current interactions, but also predicted accurately the nature of neutral-current interactions. (Given space constraints, I see no reason to go into much of the details of the theory, or the many experimental tests of its structure.)
\end{itemize}
[4 pages]
\section{Neutrino Oscillations and Neutrino Masses}
\subsection{The Evidence for Neutrino Oscillations}\label{sec:nu_osc_evidence}
\begin{itemize}
    \item Describe status quo ante of massless nature of neutrinos: BEH mechanism as exists cannot allow for neutrinos to have mass as only left-handed neutrinos have been observed.
    \item Furthermore, strong experimental limits on neutrino masses, from e.g. tritium-decay endpoint measurements by the KATRIN experiment and cosmological inferences from the CMB by the Planck satellite.
    \item But --- then neutrino oscillations are observed over a variety of experiments and contexts. Summarise critical bits of evidence:
    \item Electron neutrino disappearance in solar neutrino experiments, including Ray Davis' Homestake experiment, the SAGE/GALLEX experiments, and SNO. For the latter, the comparison of charged-current and neutral-current modes of interaction was clear evidence of neutrino oscillations over other types of process (e.g. neutrino decay).
    \item Include in the above a brief description of Bahcall's Standard Solar Model.
    \item Muon neutrino disappearance in atmospheric and long-baseline accelerator neutrino experiments, such as Super-Kamiokande, T2K, and No$\nu$a.
    \item A few further observations to note are: reactor electron anti-neutrino disappearance from both KamLAND and Daya Bay; tau neutrino appearance at the OPERA experiment; short-baseline neutrino anomaly within LSND and MiniBooNE (with recent contrary evidence from MicroBooNE).
\end{itemize}
[5 pages]
\subsection{The Phenomenology of Neutrino Oscillations}
\begin{itemize}
    \item Describe the current phenomenological model of 3-flavour neutrino oscillations that can explain all of this evidence: the PMNS mixing matrix.
    \item Describe also the MSW effect, which is critical for explaining solar neutrino oscillations.
    \item Show the formula for solar neutrino oscillations, given this MSW effect in both the Sun and Earth. Note the dependence of solar neutrino oscillations on only the ``solar'' oscillation parameters. This is all particularly useful for the solar analysis chapter.
\end{itemize}
\begin{equation}\label{eq:pee_msw}
    P_{ee}\left(\tan2\theta^{M}_{12}, \sin\theta^{M}_{13}, \Delta m^{2}_{21,M}\right) = BLAH
\end{equation}
[3 pages]
\subsection{The Origins of Neutrino Mass}
\begin{itemize}
    \item Observed neutrino oscillations require at least two neutrino mass states to be non-zero. Given constraints of the current SM, two main ways of adding neutrino masses: a Dirac mass term (i.e. allowing for sterile neutrinos), and a Majorana mass term.
    \item For latter, briefly describe what a Majorana particle is, and how with the Seesaw Mechanism (just the simple Type 1 described in-text) one can not only get neutrino masses but also explain their lightness relative to the other massive SM particles. Note that there exist more elaborate versions of this theory.
    \item Furthermore, with reference to the Sakharov conditions, describe qualitatively how the Seesaw Mechanism also allows for possible leptogenesis/baryogenesis in the early Universe, and hence could explain its matter-antimatter asymmetry.
    \item Describe briefly the nuclear physics behind double-beta decay (i.e. why it can happen at all over just normal beta decay), and then how Majorana neutrinos allow for neutrinoless double beta decay, \onbb{}.
    \item Describe the experimental signature of \onbb{}: a spike of events of observed energy equal to the Q-value of the decay.
    \item Note Schecter-Valle Theorem ensures that any observation of \onbb{} must be the result of neutrinos being Majorana. I.e. the Universe cannot conspire against us and have \onbb{} without Majorana neutrinos.
    \item Very briefly note the current status of the search for \onbb{}, describing the main varieties of experimental setup seen, along with a nice canonical example of such an experiment and their best limit. In particular, the Germanium-crystal detectors such as GERDA, Xenon-TPC detectors like EXO-200, and large-scale liquid scintillators such as KamLAND-Zen.
\end{itemize}
[3 pages]

[CHAPTER TOTAL: 17 pages]
}