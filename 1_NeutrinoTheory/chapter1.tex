%!TEX root = ../thesis.tex
%*******************************************************************************
%*********************************** First Chapter *****************************
%*******************************************************************************

\chapter{The Theory of Neutrino Physics}\label{chap:theory}
\setlength{\epigraphwidth}{.45\textwidth}
\epigraph{\textit{Light\\Light\\The visible reminder of Invisible Light}}{\textit{The Rock}\\\textsc{T. S. Eliot}}
\setlength{\epigraphwidth}{.4\textwidth}
\section{The Standard Model and Neutrinos}
\begin{itemize}
    \item First up, a very brief summary of what the Standard Model of Particle Physics is, and how it works at a high level (QFT + non-Abelian gauge theories + Brout-Englert-Higgs Mechanism). This gives the highest-level context.
    \item Next: how neutrinos fit into this picture of the SM: the currently-known terms of the SM Lagrangian containing neutrinos. From this, the interactions of neutrinos with the weak force are derived. Experimental evidence supporting this model is given. This gives us the fundamental mechanism by which neutrinos can be detected. This leads to an explanation of neutrino detection modes: inverse-beta decay, electron-neutrino elastic scattering, quasi-elastic nuclear scattering, deep inelastic scattering, and even coherent neutrino-nuclear scattering.
\end{itemize}
\section{Neutrino Oscillations and Neutrino Masses}
\begin{itemize}
    \item One major assumption underlying the current SM is that neutrinos are massless --- I'll explain why. But, neutrino oscillations have been observed! Give a brief summary of the various places where neutrino oscillations have been observed. This means we need to add two things: a mechanism for neutrinos having a mass, and a mechanism for oscillations given those masses. Starting with the former, I will cover briefly how the see-saw mechanism allows for neutrino masses, as well as the differences between Dirac and Majorana neutrinos. This also has implications for the problem of matter-antimatter asymmetry in the universe.
    \item This leads on to a description of neutrino double beta decay: what it is, how it can be searched for, and how a discovery of this decay guarantees that neutrinos have a Majorana mass term. What is the current status of research in this area (briefly)? This provides the necessary context for SNO+ as an experiment designed to search for \onbb{}.
    \item Now onto explaining the mechanism for neutrino oscillations. Describe the current phenomenological model we use: the PMNS matrix with the MSW effect in the LMA regime. Experiments that have best-constrained oscillation params: long-baseline; short-baseline; atmospheric; and, of course, solar.
\end{itemize}
\section{Solar Neutrino Oscillations}
\begin{itemize}
    \item Given the focus of my thesis on solar neutrino oscillations, give a brief summary of the history of solar neutrino oscillations. Include the added theory necessary to understand solar neutrino oscillation analyses specifically: the Standard Solar Model, and how the "solar oscillation parameters" are the only neutrino oscillation parameters that actually impact solar neutrino analyses. Also cover how, as a sub-dominant effect, day-night asymmetries can exist for solar neutrinos because of matter effects through the Earth.
    \item Cover briefly how the same solar oscillation parameters can also be measured by reactor anti-neutrino analyses, and therefore how this can provide complementary information to a solar oscillation analysis.
\end{itemize}