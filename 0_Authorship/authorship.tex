\chapter*{Statement of Authorship}
The nature of contemporary experimental particle physics is that the research work is highly collaborative. The SNO+ experiment, which this thesis is about, is an 80+ member Collaboration and as such is no exception. As a result, the work discussed in this thesis is in many places built upon the tireless work of many others within the Collaboration. This Statement, as well as throughout the thesis itself, endeavours to note where the author himself did the research, and where work was taken from others.

Chapters~\ref{chap:theory} and \ref{chap:detector} cover the background knowledge of neutrino physics as well as the SNO+ detector needed to best understand the work performed in the rest of this thesis. This information has been compiled from a variety of books, journal articles, theses, and SNO+ internal technical reports. The author was trained as both a detector `operator' and `expert', taking dozens of shifts to monitor the detector whilst it was taking data, as well as being on-call to help if any issues arose with the detector electronics. The author also made numerous contributions to the Collaboration's central software package, \texttt{RAT}.

Chapters~\ref{chap:smellie_hardware}--\ref{chap:smellie_analysis} cover the work done by the author on the SMELLIE optical calibration system for SNO+. This calibration system historically been built and analysed by multiple people, including: Krishanu Majumdar, Esther Turner, Stephanie Langrock, and Jeff Lidgard. The author installed multiple pieces of new hardware for SMELLIE on-site, with help from Armin Reichold and Jeff Lidgard. With the same team, the author also helped to integrate the new hardware with the existing SMELLIE server software so that the newly-installed hardware could actually be effectively used. The author has taken dozens of hours of calibration data using SMELLIE throughout the course of the DPhil, both by himself and alongside Armin Reichold, Jeff Lidgard, and Ana Sofia In\'{a}cio.

Chapter~\ref{chap:beam_profiling} considers improvements to the simulation of SMELLIE events; the work done by Esther Turner on this subject is considered as a starting point. The rest of the chapter covers a new simulation approach created and developed by the author. Similarly, the scattering analysis performed within Chapter~\ref{chap:smellie_analysis} was inspired from the analyses done by Krishanu Majumdar, Esther Turner, and Stephanie Langrock. However, the work in this thesis uses a new method created by the author, and uses new scintillator phase data taken by the author. Also in this chapter is a separate analysis of this same data to measure the scintillator's extinction length. This was a novel analysis for SMELLIE, designed and implemented entirely by the author.

Finally, Chapter~\ref{chap:solar_osc_analysis} describes an analysis of scintillator phase data performed by the author to measure the solar neutrino oscillation parameters. An initial background-free sensitivity study of this topic was originally performed by Javier Caravaca. However, the work done by the author in this thesis analyses actual data, includes all relevant backgrounds appropriately, and considers systematics. Projections of expected sensitivity with greater livetime were also performed by the author. The analysis made by the author also uses a Bayesian MCMC approach, using the \texttt{OXO} signal extraction software framework initially built by Jack Dunger. A number of further people have since improved \texttt{OXO}, including the author: substantial improvements to the handling of systematics floated within the MCMC fit, as well as allowing for the floating of oscillation parameters within the fit.