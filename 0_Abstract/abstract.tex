% ************************** Thesis Abstract *****************************
% Use `abstract' as an option in the document class to print only the titlepage and the abstract.
\begin{abstract}
SNO+ is a large-scale liquid scintillator experiment based in Sudbury, Canada, capable of probing many aspects of neutrinos. One major property of interest is the neutrino's ability to oscillate between different flavours, an indirect demonstration that neutrinos must have mass.

This thesis performs the first ever measurement of oscillations from \beight{} solar neutrinos in the scintillator phase of SNO+. Assuming the current global fit flux of \beight{} solar neutrinos, the neutrino oscillation parameter \tonetwo{} was measured to be $38.9^{\circ+8.0^{\circ}}_{\phantom{\circ}-7.9^{\circ}}$, using an initial 80.6 days of data. This result is consistent with the current global fit result of $33.44^{\circ+0.77^{\circ}}_{\phantom{\circ}-0.74^{\circ}}$. A sensitivity study indicates that the precision of this result is capable of improving by at least a factor of two within two years of livetime.

On top of this, substantial improvements were made to all aspects of the optical calibration system known as SMELLIE. This is a series of optical-wavelength lasers whose light is emitted from optical fibres attached to the edge of the SNO+ detector.  By developing a new analysis, this system was able to measure the scintillator extinction lengths as a function of wavelength and time \textit{in-situ} for the first time. A new analysis was also built and demonstrated to observe changes in scattering and scintillator re-emission properties of the scintillator as a function of time and wavelength. Alongside this, major upgrades were made to both the hardware and simulation of the SMELLIE system, enabling higher-quality data to be taken, and simulations to be made with much greater speed.
\end{abstract}