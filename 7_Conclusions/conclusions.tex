\chapter{Conclusions}\label{chap:conclusions}
{
    \color{blue}
    Say what has been achieved in this thesis! In particular:
\begin{itemize}
    \item Substantial improvement to the SMELLIE generator in terms of speed and dynamic range
    \item A much stronger understanding of the discrepancies between data and MC in SMELLIE
    \item The creation of two analyses of SMELLIE data, designed explicitly around being robust to these systematics
    \item A measurement of the extinction length of scintillator \textit{in-situ} with SMELLIE at \SI{375}{\nano\metre}, monitored over time
    \item A first measurement of the scattering length of the scintillator \textit{in-situ}, monitored over time
    \item The creation of an analysis of \beight{} solar neutrinos in the scintillator phase to measure the solar neutrino oscillation parameters
    \item The first measurement of \tonetwo{} using \beight{} neutrinos in SNO+
    \item Projections of this solar analysis' precision at longer livetimes
    % \item Give suggestions for further work that could be done on both SMELLIE and the solar oscillation analysis:
    % \begin{itemize}
    %     \item Inclusion of LABPPO's polarisability anisotropy in the detector's optical scattering simulation, and determination of its impact on both Physics and SMELLIE measurements
    %     \item Further investigation of SMELLIE's wavelength-dependence of the beam profiles, looking both at possible origins for the phenomenon and correcting for this in simulation
    %     \item Various computational improvements to allow for faster MCMC run-times, including the calculation of the systematic matrices, and possible use of GPUs to parallelise some parts of the computation.
    %     \item Inclusion of additional solar neutrino components at lower energy into the fit, e.g. Be7. Maybe the addition of lower energies also helps to naturally constrain backgrounds within the fit?
    %     \item Looking at the impact of various advanced background-rejection procedures, such as event directionality or topology. How much do they help with the sensitivity?
    %     \item Looking at the impact of splitting data into day and night parts, to try and provide further constraints on any matter effects. Not expected to be significant, so was ignored for this analysis so far. 
    %     \item Performing a combined fit with the reactor anti-neutrino oscillation analysis, which allows for the handling of correlated uncertainties, such as detector response systematics.
    % \end{itemize}
\end{itemize}
[3 PAGES TOTAL]
}