\chapter{Conclusions}\label{chap:conclusions}
\setlength{\epigraphwidth}{.8\textwidth}
\epigraph{\textit{I believe we've reached the end of our journey.\\All that remains is to collapse the innumerable possibilities before us.\\Are you ready to learn what comes next?}}{\textsc{Solanum}, \textit{Outer Wilds}}
This thesis covered two major aspects of the SNO+ experiment during the scintillator phase. One of these was the first ever neutrino oscillation analysis built and performed on solar neutrino data from the scintillator phase. Using 80.6 days of livetime, the neutrino oscillation parameter \tonetwo{} was measured to be $38.9^{\circ+8.0^{\circ}}_{\phantom{\circ}-7.9^{\circ}}$, assuming the global fit measurement of the \beight{} flux, $\Phi_{\beight{}} = (5.16\,^{+2.5\%}_{-1.7\%})\times 10^{6}\,\si{\per\cm\squared\per\second}$. In order to make this measurement, a sophisticated Bayesian Analysis framework was developed, using MCMC. This also required building a model of the various background events present in the data, and calibrate the energy scale systematic parameter.

This oscillation analysis is limited by the statistics of the \beight{} neutrinos. A sensitivity study showed that, assuming the same conditions seen in the dataset analysed, the measurement precision can be improved by a factor of two in under two years of livetime. There is some scope for improvement to this sensitivity with the addition of bis-MSB to the detector, as well as optimisations to the analysis described in Section~\ref{sec:solar_summary}. It may be possible to provide a much stronger result if combined with the reactor anti-neutrino oscillation analysis also being performed with SNO+.

The other major focus of this thesis has been the optical calibration system SMELLIE. Major hardware upgrades were made, improving the stability of the PQ lasers' emission intensities, the triggering system used for SMELLIE, and the Monitoring PMT Unit. A new algorithm for SMELLIE event generation was developed, increasing the speed of simulations by many orders of magnitude. A method for combining SMELLIE data to determine the beam profiles was built and implemented.

One consequence of the building of these new beam profiles is the more precise understanding of existing systematics in the optical model of \texttt{RAT} and SMELLIE. To still make progress despite these issues, a new set of analyses for SMELLIE were developed. One compared water phase and scintillator phase data to measure the extinction length of scintillator as a function of both wavelength and time, and another compared changes in scattering and re-emitted scintillation light. From these measurements, there is evidence of an optical component added to the scintillator between May 2021 and May 2022 which shortens the observed extinction lengths measured in the \SIrange{400}{500}{\nm} range, but decreases the total amount of scattering and re-emitted light. There does not appear to be any substantial change to the observed optics following this.

It is an exciting time for the SNO+ experiment: multiple interesting analyses of scintillator phase data are underway, with calibrations allowing us to get an increasingly more accurate model of the detector. In the near future, tellurium will be loaded into the scintillator cocktail, allowing for the \onbb{} campaign to begin in earnest. Only time will tell how successful this will be!

% {
%     \color{blue}
%     Say what has been achieved in this thesis! In particular:
% \begin{itemize}
%     \item Substantial improvement to the SMELLIE generator in terms of speed and dynamic range
%     \item A much stronger understanding of the discrepancies between data and MC in SMELLIE
%     \item The creation of two analyses of SMELLIE data, designed explicitly around being robust to these systematics
%     \item A measurement of the extinction length of scintillator \textit{in-situ} with SMELLIE at \SI{375}{\nano\metre}, monitored over time
%     \item A first measurement of the scattering length of the scintillator \textit{in-situ}, monitored over time
%     \item The creation of an analysis of \beight{} solar neutrinos in the scintillator phase to measure the solar neutrino oscillation parameters
%     \item The first measurement of \tonetwo{} using \beight{} neutrinos in SNO+
%     \item Projections of this solar analysis' precision at longer livetimes
%     % \item Give suggestions for further work that could be done on both SMELLIE and the solar oscillation analysis:
%     % \begin{itemize}
%     %     \item Inclusion of LABPPO's polarisability anisotropy in the detector's optical scattering simulation, and determination of its impact on both Physics and SMELLIE measurements
%     %     \item Further investigation of SMELLIE's wavelength-dependence of the beam profiles, looking both at possible origins for the phenomenon and correcting for this in simulation
%     %     \item Various computational improvements to allow for faster MCMC run-times, including the calculation of the systematic matrices, and possible use of GPUs to parallelise some parts of the computation.
%     %     \item Inclusion of additional solar neutrino components at lower energy into the fit, e.g. Be7. Maybe the addition of lower energies also helps to naturally constrain backgrounds within the fit?
%     %     \item Looking at the impact of various advanced background-rejection procedures, such as event directionality or topology. How much do they help with the sensitivity?
%     %     \item Looking at the impact of splitting data into day and night parts, to try and provide further constraints on any matter effects. Not expected to be significant, so was ignored for this analysis so far. 
%     %     \item Performing a combined fit with the reactor anti-neutrino oscillation analysis, which allows for the handling of correlated uncertainties, such as detector response systematics.
%     % \end{itemize}
% \end{itemize}
% [3 PAGES TOTAL]
% }