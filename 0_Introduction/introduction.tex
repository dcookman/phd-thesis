\chapter*{Introduction}\label{chap:intro}
It is an exciting time in Neutrino Physics. Over the past decades, the evidence for neutrinos oscillating between different flavours has become overwhelming, to the point that the Nobel Prize in Physics was awarded in 2015 for the discovery~\cite{NobelPrizePhysics2015}. However, as one set of questions gets answered, others get raised: if neutrinos oscillate as they propagate through space, then by Special Relativity they cannot be massless. But what are those masses? What underlying mechanism enables them to be massive, but still requires them to be extraordinarily light? And can we make precision measurements of how this oscillation phenomenon occurs?

These are some of the major questions that current neutrino physics experiments seek to answer. One such experiment is SNO+, a large-scale multipurpose neutrino detector built \SI{2}{\km} underground in Sudbury, Canada. Filled with 800 tonnes of liquid scintillator, neutrinos from a wide variety of sources can be detected. The primary goal is the search for neutrinoless double beta decay (\onbb{}): if discovered, it would be smoking-gun evidence of how neutrinos get mass, and could also provide us a way of measuring the neutrino mass scale.

This thesis describes work done by the author to help progress the SNO+ experiment. These efforts can be split into two categories: optical calibration of the detector, and performing an oscillation analysis of solar neutrinos. In order to provide sufficient context for these results, this document begins firstly with a chapter summarising important results in neutrino physics, followed by a chapter on the details of the SNO+ detector.

Chapters 3--5 cover the work done on one of the optical calibration systems for SNO+, known as SMELLIE. The system is introduced in Chapter~\ref{chap:smellie_hardware}, with explanations of how the system works, and the hardware upgrades made in the Summer of 2022. Chapter~\ref{chap:beam_profiling} explains how improvements were made to the simulation of SMELLIE events, including a dramatic reduction in the time needed to generate a simulated event. There is also a discussion of the remaining discrepancies that still exist between data and simulation, despite improvements made to the simulation of the angular emission distributions. With these systematic effects in mind, Chapter~\ref{chap:smellie_analysis} goes over the creation and implementation of two new analyses of SMELLIE data. In one, the extinction lengths of the scintillator were measured \textit{in-situ} as a function both of wavelength and time. In the other, changes in the scattering length and scintillation re-emission properties could be observed. Understanding the optical properties of the detector with precision is critical to being able to perform physics analyses with data from SNO+.

One of the major analyses of interest (other than \onbb{}) for SNO+ is measuring the parameters that govern neutrino oscillations. This can be achieved by looking at the observed energy spectrum of \beight{} solar neutrinos: this is the subject of Chapter~\ref{chap:solar_osc_analysis}. For the first time, an analysis has been built and performed using scintillator phase data from the detector. The future prospects of this analysis as more data is taken is also investigated.